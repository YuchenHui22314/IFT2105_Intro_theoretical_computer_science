\documentclass{article}

%========================================
\usepackage{tabularx, makecell, multirow }
\usepackage{syntonly}
% \usepackage{hyperref}
%utf8
\usepackage[utf8]{inputenc}
\usepackage[bottom]{footmisc}
\usepackage{array}
\usepackage{booktabs}
\usepackage{graphicx}
\usepackage{amsmath}
\usepackage{amsthm}
\usepackage{enumitem}



%========================================

%----------------- new command ----------%            -{}-
\newcommand{\ac}[1]{\left\{ #1 \right\}}

\title{\textbf{IFT2105 Devoir4}}
\author{
    Samy Rasmy 20214818 \and Yuchen Hui 20150470}
\date{\today}

\begin{document}

\maketitle

\section*{Question1}
On va montrer dans deux sens.
\subsection*{$f$ est calculabe $\implies L_{f} \text{ est décidable}$ }
\begin{proof}
	On va prover par construire une machine de Turing $M$ qui décide $L_{f}$ tout en tirant avantage du fait qu'il existe une machine de Turing $M'$ qui, exécuté sur tout entrée $x \in \Sigma^{*}$, s'arrête avec $f\left( x \right) $ sur le ruban:\\
M = `` Sur l'entrée <x,y>: 
	\begin{enumerate}
		\item Simuler $M'$ sur x. (NB $M'$ s'arrêtera sur tout entrée x dans Sigma étoile.)   
		\item Comparer $f\left( x \right) $ calculée par $M'$ avec y. S'il sont égales, \textit{accepter}; Sinon \textit{rejeter}."
	\end{enumerate}
Ça se voit que  $M$ décide le langage $L_{f} = \left\{ \text{<}x,y\text{>}\ |\  y = f\left( x \right)  \right\} $.
\end{proof}

\subsection*{$L_{f} \text{ est décidable}$ $\implies f \text{ est calculable}$}
\begin{proof}
	Soit $M'$ la machine de Turing qui décide $L_{f}$, soit $s_1, s_2, s_3\ldots$ une liste de tous les mots possibles dans $\Sigma^{*}$ (dans un ordre quelconque). Construisons une machine de Turing M ci-dessous:\\
M = `` Sur l'entrée x: 
	\begin{enumerate}
		\item Répéter les étapes ci-dessous pour $i = 1,2,3,\ldots$ 

		\item \qquad Simuler $M'$ sur l'entrée <$x,s_{i}$>. 
		\item \qquad Si $M'$ accepte, mettre $f\left( x \right) $ à la fin du ruban et mettre la tête sur le premier symbole du $f \left( x \right) $, puis \textit{accepter}, si $M'$ rejette, continuer avec la prochaine i."
	\end{enumerate}
On voit bien que $M$, exécuté sur tout entrée $x \in \Sigma^{*}$, s'arrête avec $f\left( x \right) $ sur le ruban.
\end{proof}
\section*{Question2}
On va montrer dans deux sens.
\begin{proof}
	D'abord, on montre que s'il existe un énumérateur $E$ qui énumère langage $L$, il exsite une machine de Turing $M$ qui reconnait $L$. La définition de $M$ est donnée ci-dessous:\\
M = `` Sur l'entrée $w$: 
	\begin{enumerate}
		\item Simuler $E$. Chaque fois que E imprime un mot, le comparer avec $w$. 

		\item  Si $w = \text{cet mot imprimé}$, \textit{accepter}." 
	\end{enumerate}
	On voit bien que $M$ accepte tous les mots du language $L$.
	
	Considérons maintenant l'autre direction. Supposons qu'il existe une machine de Turing M qui reconnait le language L. Soit $s_1, s_2, s_3,\ldots$ une liste de tous les mots dans $\Sigma^{*}$ (d'un ordre quelconque), on va construire un énumérateur $E$ qui imprimera éventuellement chaque mot de $L$ et seulement les mots de $L$. 
E = ``Sur toute entrée : 
\begin{itemize}
\setlength{\itemsep}{-0.5em}
	\item[] \qquad$i = 0$ 
	\item[]\qquad tant que $true$ :
\item[] \qquad\qquad $i \gets i+1$ 
\item[] \qquad\qquad pour tout k allant de $1\ldots i$ :
\item[]\qquad\qquad\qquad $w \gets  \text{le $k^{ème}$ mot dans $\Sigma^{*}$ en ordre lexicographique }$ 
\item[] \qquad\qquad\qquad Simuler M sur w pour i étapes
\item[] \qquad\qquad\qquad si $M$ accepte $w$, imprimer le mot $w$."

\end{itemize}
Cet énumérateur imprimera éventuellement tous les mots dans le language $L$, car il ne va pas boucler infiniment sur des mots qui n'appartiennent pas au $L$.
\end{proof}
\section*{Question3}
Considérons le langage $\overline{FINI_{TM}}$, soit $$\left\{ \text{<}M\text{>} \ |\ M \text{ une machine de Turing et L(M) infini}\right\}. $$ Si on arrive à prouver qu'il est indécidable, $FINI_{TM}$ est automitiquement indécidable. Soit la fonction de réduction du language $A_{MT}$ au langage $\overline{FINI_{MT}}$:
\begin{align*}
	f: \Sigma^{*} &\longrightarrow \Sigma^{*} \\
	 \text{<}M,w\text{>}  &\longmapsto f\left( \text{<}M,w\text{>} \right)= \text{<}M'\text{>}
,\end{align*}
telle que
$M'$ = ``Sur entrée x: 
\begin{enumerate}
\setlength{\itemsep}{-0.5em}
\item  Simuler M sur w. S'il rejette, \textit{rejeter}. S'il accepte, aller à l'étape 2.
\item Accepter n'importe quel entrée $x$. 
\end{enumerate}
On voit bien que
\begin{align*}
	<M,w> \notin  A_{TM} & \implies \text{M va rejeter w ou boucler infiniment sur w}\\
			& \implies M' \text{rejettera x ou boucler infiniment sur x}\\
			& \implies L(M') = \phi ,\text{qui n'est pas infini}\\
			& \implies \text{<}M'\text{>} \not\in \overline{FINI_{MT}}
.\end{align*}
D'autre part,
\begin{align*}
	<M,w> \in  A_{TM} & \implies \text{M acceptera w}\\
			& \implies M' \text{ acceptera n'importe quel x}\\
			& \implies L(M') = \Sigma^{*} ,\text{qui est infini}\\
			& \implies \text{<}M'\text{>} \in \overline{FINI_{MT}}
.\end{align*}
Puisque $A_{TM}$ est indécidable, on peut conclure que $\overline{FINI_{MT}}$ est indécidable. En conséquence, le language $FINI_{MT}$ est indécidable.


\end{document}

