

\documentclass{article}

%========================================
\usepackage{tabularx, makecell, multirow }
\usepackage{syntonly}
\usepackage{hyperref}
%utf8
\usepackage[utf8]{inputenc}
\usepackage[bottom]{footmisc}
\usepackage{array}
\usepackage{booktabs}
\usepackage{graphicx}
\usepackage{amsmath}
\usepackage[]{amsthm}
%========================================

%----------------- new commands ----------
%            -{}-
\newcommand{\ac}[1]{\left\{ #1 \right\}}

\title{\textbf{IFT2105 Devoir2}}
\author{
    Samy Rasmy 20214818 \and Yuchen Hui 20150470}
\date{\today}


\begin{document}
\maketitle
\section*{Avant tout}
On va d'abord présenter sous forme de la logique propositionnell les deux lemmes du pompistes et respectivement leurs contraposées, car on prouvera nos conclusions en utilisant les deux contraposées.
 

Premièrement, lemme du pompiste régulier:

\begin{align*}
&	L \in REG \implies \\
&	\exists p \ge 1 ,\forall w \left( w \in L \right) \wedge \left(\left| w \right| \ge p \right) \exists x,y,z \in \Sigma^{*}\\
&	\left [ \left( w=xyz \right) \wedge \left( \left| y \right| \ge 1 \right) \wedge \left(\left| xy \right| \le p\right) \wedge \left( \forall  i \ge 0, xy^{i}z \in L \right) \right] 
\end{align*}
et sa contreposée: 
\begin{align*}
&	\forall p \ge 1 ,\exists w \left( w \in L \right) \wedge \left(\left| w \right| \ge p \right) \forall x,y,z \in \Sigma^{*}\\
&\left [ \left( w=xyz \right) \wedge \left( \left| y \right| \ge 1 \right) \wedge \left(\left| xy \right| \le p\right) \implies \left( \exists  i \ge 0, xy^{i}z \not\in L \right) \right] \\
&	\implies L \not\in REG.
\end{align*}

Deuxièmement, lemme du pompiste HC:
\begin{align*}
&	L \in HC \implies \\
&	\exists p \ge 1 ,\forall w \left( w \in L \right) \wedge \left(\left| w \right| \ge p \right) \exists u,v,x,y,z \in \Sigma^{*}\\
&	\left [ \left( w=uvxyz \right) \wedge \left( \left| vy \right| \ge 1 \right) \wedge \left(\left| vxy \right| \le p\right) \wedge \left( \forall  i \ge 0, uv^{i}xy^{i}z \in L \right) \right] 
\end{align*}
et sa contreposée: 
\begin{align*}
&	\forall p \ge 1 ,\exists w \left( w \in L \right) \wedge \left(\left| w \right| \ge p \right) \forall u,v,x,y,z \in \Sigma^{*}\\
&\left [ \left( w=uvxyz \right) \wedge \left( \left| vy \right| \ge 1 \right) \wedge \left(\left| vxy \right| \le p\right) \implies \left( \exists  i \ge 0, uv^{i}xy^{i}z \not\in L \right) \right] \\
&	\implies L \not\in HC.
\end{align*}
\clearpage
\section*{Question1}
\textbf{Conclusion}: le langage est hors-context.\\
\textbf{Grammaire}: $S\to SS\ |\ aSbSa\ |\ bSaSa\ |\ aSaSb\ |\ \epsilon$.\\
Maintenant on prouve que le langage n'est pas régulier:
\begin{proof}
	Soit p un entier positif arbitraire, considérons le mot $w = a^{2p}b^{p} \in L$. On a bien que $ |w| = 3p > p$.
	Soit $xyz = w$ une décomposition quelconque en trois partie de w. Supposons que les deux premières conditions du lemme tiennent, soient $|xy| \le  p$ et $|y| \ge  1$. Cherchons un entier positif $i$ tel que $xy^{i}z \not\in L$: Puisque $ |xy| \le p \text{ et } |y| \ge 1$, y ne contient que le symbole a, alors le mot $xy^{2}z$ prendra la forme $a^{2p+ |y|}b^{p}$. On voit bien que $2p + |y| \neq 2\cdot p$, d'où le mot $xy^{2}z \not\in L$. D'après la contraposée du lemme du pompiste régulier, $L \not\in REG$.
\end{proof}
\section*{Question2}
\textbf{Conclusion}: le langage n'est pas hors-context.\\
\begin{proof}
Soit p arbitraire et le mot $a^{p^{2}}b^{p}$.\\
Soit une décomposition quelconque: $w = uvxyz$, dont $|vy| \ge 1.$ Cherchons des i tel que $uv^{i}xy^{i}z \not\in L.$\\
Soit B le nombre de b dans vy, et A le nombre de a dans vy.\\
Le nombre de a au total dans le mot pompé sera $p^{2}+Ai$. Celui de B sera $p+Bi$.\\ 
La somme A+B ne peut pas etre 0. Au moins un de ces deux nombres sera donc plus grand ou egal a 1.\\
Nous pouvons reecrire la regle de la grammaire de $|w|_{a} = \left( |w|_{b} \right)^{2}$ vers $p^{2} + Ai = p^2 + 2Bip + (Bi)^2$.\\
Nous pouvons reecrire tel que:
\[
2Bip+(Bi)^2-Ai= 0
.\] 
p etant constant.\\
\textbf{Si $A = 0$ et $B!=0$}, cela revient à $2Bip + \left( Bi \right)^{2} = 0 $.\\
Ceci n'est pas possible pour tout i entier positif.\\
\textbf{Si $B=0$ et $A!=0$}, cela revient à $Ai = 0$.\\
Pour les memes raisons, ceci ne peut etre vrai pour tout i.\\
\textbf{Si $B!=0$ et $A!=0$}, cela revient à $2Bp = A-B^2i$.\\
Le coté droit étant une fonction strictement décroissante alors que le coté gauche est constant, ceci ne peut être vrai pour tout i.\\
En conclusion, dans tours les cas on est capable de trouver un $i$ entier positif tel que $uv^{i}xy^{i}z \not\in L$.
Donc, le language n'est pas hors contexte. 
\end{proof}
\section*{Question3}
\textbf{Conclusion}: le langage est hors-context.\\
\textbf{Grammaire}:
\begin{align*}
	S &\to aSc\ |\ BC\\
	BC&\to bBCc\ |\ \epsilon
.\end{align*}
Maintenant on prouve que le langage n'est pas régulier:
\begin{proof}
	Soit p un entier positif arbitraire, considérons le mot $w = a^{p}b^{p}c^{2p} \in L$. On a bien que $ |w| = 4p > p$.
	Soit $xyz = w$ une décomposition quelconque en trois partie de w. Supposons que les deux premières conditions du lemme tiennent, soient $|xy| \le  p$ et $|y| \ge  1$. Cherchons un entier positif $i$ tel que $xy^{i}z \not\in L$: Puisque $ |xy| \le p \text{ et } |y| \ge 1$, $y$ ne contient que des symboles a, alors le mot $xy^{2}z$ prendra la forme $a^{p+ |y|}b^{p}c^{2p}$. On voit bien que $|w|_{a}+|w|_{b} = (p + |y|) + p = 2p + |y|\neq 2p =  |w|_{c}$, d'où le mot $xy^{2}z \not\in L$. D'après la contraposée du lemme du pompiste régulier, $L \not\in REG$.
\end{proof}
\section*{Question4}
\textbf{Conclusion}: Le langage est hors-context.\\
\textbf{Grammaire}:
\begin{align*}
	S &\to aSc\ |\ B\\
	B&\to bB\ |\ \epsilon
.\end{align*}
Maintenant on prouve que le langage n'est pas régulier:
\begin{proof}
	Soit p un entier positif arbitraire, considérons le mot $w = a^{p}b^{p}c^{p} \in L$. On a bien que $ |w| = 3p > p$.
	Soit $xyz = w$ une décomposition quelconque en trois partie de w. Supposons que les deux premières conditions du lemme tiennent, soient $|xy| \le  p$ et $|y| \ge  1$. Cherchons un entier positif $i$ tel que $xy^{i}z \not\in L$: Puisque $ |xy| \le p \text{ et } |y| \ge 1$, $y$ ne contient que des symboles a, alors le mot $xy^{2}z$ prendra la forme $a^{p+ |y|}b^{p}c^{p}$. On voit bien que $|w|_{a} = (p + |y|) \neq p = |w|_{c}$, d'où le mot $xy^{2}z \not\in L$. D'après la contraposée du lemme du pompiste régulier, $L \not\in REG$.
\end{proof}
\section*{Question5}

\end{document}
